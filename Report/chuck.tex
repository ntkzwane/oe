We made use of the OWL API written in Java to access the contents of the ontology. Some of the key problems we had with the API is that there was very little support online, an queries to problems sometimes came back unanswered; the documentation was not updated, so we had to infer ourselves how the updated syntax of the API class functions is structured.\\
\\
After finally working out how to correctly use the API, we continued to begin the process of verbalizing isiZulu. The language grammar has properties that makes it fit into the Bantu Languages ontology occupying a subset of the various noun classes\footnote{The ontology used, created by Catherine Chavula contains the noun classes we required, available at http://www.meteck.org/files/ontologies/ncs.owl}. Building a template for this is seemingly unrealistic, as it will not accurately capture all (or at least most) of the properties that the language's grammar possesses. There has been work on a resolve, however, by Dr. M Keet and Dr L Khumalo et al, where they worked on an algorithm to join the noun suffixes with their corresponding prefix depending on the axiom being asserted. We decided to further their work by implementing their algorithms, namely Algorithm 1 and Algorithm 2\footnote{Basica for a grammar engine to verbalize logical theories in isiZulu}.\\
\\
